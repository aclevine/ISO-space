\documentclass{article}
\usepackage[utf8]{inputenc}
\usepackage[T1]{fontenc}
\usepackage{geometry}
\usepackage{graphicx}
\usepackage{amsmath}
\usepackage{amssymb}
\usepackage{textcomp}
\usepackage{tikz}
\geometry{a4paper}
\usepackage[francais]{babel}
\title{Translating SPRL to ISO-Space}
\author{Seth Dworman}
\date{16 September 2014}

\begin{document}
\maketitle
$_{}$
\def\checkmark{\tikz\fill[scale=0.4](0,.35) -- (.25,0) -- (1,.7) -- (.25,.15) -- cycle;}
\newenvironment{attributes}
{\medskip\medskip
 \begin{tabular}{|l|l|}
 \hline} 
{\hline
 \end{tabular}
 \medskip\medskip}
\\
{\large {\bf 1.0 Task Description}}
\\
\\
The current task is to find a way to translate SPRL (spatial role labeling) into the ISO-Space annotation schema.  I have done little research as to whether this sort of annotation scheme translation task has be done before.  The likelihood is it probably has not, as most annotation schemes go through a MAMA cycle, and therefore there is generally some kind of isomorphism between different \emph{dialects} of the same annotation mark up language.  Thus this case is different in that SPRL and ISO-Space are not different versions of the same underlying annotation language but instead separate \emph{languages}.  What makes a translation possible is that both of them attempt to model the same information: spatial relations in natural language text (and images).  Finding an effective mapping between SPRL and ISO-Space would be useful if there were a large number of documents annotated in the (now deprecated?) SPRL language, but which had not yet been annotated for ISO-Space.  This would provide more training data.  The question is, can this be done, and if so, how?  
\\
\\
One way to approach translating SPRL into ISO-Space is to view the SPRL tags as not necessarily isomorphic to those of an ISO-Space annotation on the same document, but rather as simply providing additional information that would prove useful in attempting automatic machine learning based annotation of a document.  Assuming that the SPRL annotated documents are correctly annotated, it should be {\bf easier} to generate ISO-Space tags from an SPRL document than a completely unannotated document (which is one of the goals of ISO-Space).  Thus this task is not fundamentally different from the general problem at hand: given an unannotated document, generate the ISO-Space tags.  Given this, one assumption is that the extra information given by SPRL documents should make machine based annotation easier.  Thus, translating between these documents can provide a useful data point to compare with efforts on completely unannotated text.  On the other hand, a negative result, i.e. SPRL markup does not markedly improve the machine based annotation, might suggest that SPRL does not actually add a lot of useful information for determining the spatial relations, or that the information SPRL captures can be easily gathered automatically with a clever learner / algorithm.    
\\
\\
In the first section I give a brief but comprehensive description of the overall state of both SPRL tags and ISO-Space tags.  This is followed by a comparison of the two, where it shown that ISO-Space is semantically richer and more verbose, in that it captures more spatial information and requires detailed typing and specification of participants and signals, whereas SPRL is much sparser and generic in its spatial relation language.  This is important because of the implications it has for a direct (isomorphic) translation of SPRL to ISO-Space (i.e. it is not an \emph{easy} problem).  
\\
\\
{\large {\bf 2.0 SPRL}}
\\
\\
SPRL's tag set, like ISO-Space, is divided between extents and relations (or links).  Extents always have a presence in the document, in that they are tied to specific text.  Relations, on the other hand, are formed by the composition of tagged extents in the form of a $n$-tuple.  SPRL provides 7 extent tags: {\sc trajector}, {\sc landmark}, {\sc spatial\_indicator}, {\sc motion\_indicator}, {\sc path}, {\sc direction}, and {\sc distance}.  These are each described in more detail below.  
\\
\\
{\bf Trajector Tag}: The entity whose location is being described (e.g. the figure).  This tag can include agents (humans, animals), objects, and events.  The {\sc trajector} tag itself does not include any additional information, other than which text is considered a {\sc trajector}.  

\begin{attributes}
{\tt id} 			&	\texttt{T0, T1, T2},\ldots\\
\hline
{\tt text}	&	\emph{Daniel and I, we, the motorcycles},\ldots \\
\end{attributes}
\\
\\
{\bf Landmark Tag}: The entity which is being referenced to describe the location / position of a {\sc trajector} (e.g. the ground).  This can include places, paths, agents, objects, and events.  Like {\sc trajector}, {\sc landmark} does not include any information besides the text corresponding to that tag.

\begin{attributes}
{\tt id} 			&	\texttt{L0, L1, L2},\ldots\\
\hline
{\tt text}	&	\emph{Romania, the forest, the street},\ldots \\
\end{attributes}
\\
\\
{\bf Path Tag}: A schematic characterization of the minimal path (per cognitive semantics): it describes the trajector's motion in relation to a region / landmark.  It has a \emph{beginning, middle} and \emph{end}.  The tag itself only includes the text extent.  

\begin{attributes}
{\tt id} 			&	\texttt{P0, P1, P2},\ldots\\
\hline
{\tt text}	&	\emph{the forest road, through this road section, into the hardwood forest },\ldots \\
\end{attributes}
\\
\\
{\bf Spatial Indicator Tag}: Generally this correspond to a preposition that provides spatial information.  It can also include verbs, nouns, and adverbs or a combination of them.  The tag itself only includes the extent text.

\begin{attributes}
{\tt id} 			&	\texttt{S0, S1, S2},\ldots\\
\hline
{\tt text}	&	\emph{to, lie on, is },\ldots \\
\end{attributes}
\\
\\
{\bf Motion Indicator Tag}: These are mostly prepositional verbs but other categories are possible.  Eventually these should be mapped to verb motion classes.  The tag itself only includes the extent text.

\begin{attributes}
{\tt id} 			&	\texttt{M0, M1, M2},\ldots\\
\hline
{\tt text}	&	\emph{drove, bend off, take the way back},\ldots \\
\end{attributes}
\\
\\
\newpage
$_{}$
\\
{\bf Direction Tag}: Used to denote direction along some axes depending on the frame of reference.  These are used when the trajector is not in a spatial relation with an actual, overt landmark.  The tag itself only includes the extent text.

\begin{attributes}
{\tt id} 			&	\texttt{DIR0, DIR1, DIR2},\ldots\\
\hline
{\tt text}	&	\emph{southwest, right, south},\ldots \\
\end{attributes}
\\
\\
{\bf Distance Tag}: A scalar entity that can be qualitative as in \emph{close} or \emph{far away}, but also quantitative as in \emph{five miles down the road}.  This should be classified as either \emph{absolute} or \emph{relative}, but this attribute is not present in the actual tag; the tag itself only includes the extent text.

\begin{attributes}
{\tt id} 			&	\texttt{DIS0, DIS1, DIS2},\ldots\\
\hline
{\tt text}	&	\emph{some thousand off road kilometers, very near, within 530 meters},\ldots \\
\end{attributes}
\\
\\
SPRL only offers a single link, called {\sc relation}, to encode all the spatial information between the various extents  
\\
\\
{\bf Spatial Relation Link}: {\sc relation}
\\
{\sc relation}s, \emph{spatial relations}, or SRs in the most simple descriptions are composed of a {\sc trajector}, {\sc landmark}, and {\sc spatial\_indicator}, e.g. \emph{She is at school}.  They also have other fields for tags when relevant: {\tt path\_id}, {\tt direction\_id},  and {\tt motion\_id}.  These correspond to the tags described above.  Additionally, SRs also provide attributes for information about the kind of spatial relation.  This information does not always have a direct presence in the text, so in some sense it is provided by the annotator's understanding of language.  These are {\tt FoR}, {\tt relative\_value}, {\tt absolute\_value}, {\tt RCC8\_value}, {\tt specific\_type}, {\tt general\_type}, {\tt qualitative\_value}, and {\tt quantitative\_value}.  

\begin{attributes}
{\tt id} 			&	\texttt{SR0, SR1, SR2, }\ldots\\
\hline
{\tt FoR} &				{\sc relative, intrinsic, absolute} \\
\hline
{\tt relative\_value} & 		{\sc right, behind, front, below, above, other, }\ldots \\
\hline
{\tt absolute\_value} & 		{\sc north, west, sw, ne, se, east, south, nw, }\ldots \\
\hline
{\tt RCC8\_value} & 		{\sc dc, ec, po, eq, tpp, tppi, nttp, nttpi, in } \\
\hline
{\tt specific\_type} &		{\sc relative, quantitative, qualitative, rcc8, absolute} \\
\hline
{\tt general\_type} &		{\sc distance, direction, region} \\
\hline
{\tt qualitative\_value} & 	Identifier of the qualitative {\sc distance} tag used \\
\hline
{\tt quantitative\_value} & 	Identifier of the quantitative {\sc distance} tag used \\
\hline
{\tt path\_id} & 	Identifier of the {\sc path} tag used \\
\hline
{\tt direction\_id} & 	Identifier of the {\sc direction} tag used \\
\hline
{\tt motion\_id} & 	Identifier of the {\sc motion} tag used \\
\hline
{\tt spatial\_indicator\_id} & 	Identifier of the {\sc spatial\_indicator} tag used \\
\hline
{\tt landmark\_id} & 	Identifier of the {\sc landmark} tag used \\
\hline
{\tt trajector\_id} & 	Identifier of the {\sc trajector} tag used \\
\hline
\end{attributes}
%\footnote{in CP.gold only {\sc NTPP-1, NTPP, DC, EC, EQ, and PO show up}}
\newpage
$_{}$
{\large {\bf 3.0 ISO-Space}}
\\
\\
The full specification of ISO-Space will not be reproduced, since it is readily available.  Instead, each extent and link tag is summarized below in a tuple notation.    
\\
\\
{\bf Place Tag}: (\texttt{id} = [\texttt{pl0, pl1, pl2, }\ldots], \texttt{dimensionality} = [{\sc area, point, line, volume}], \texttt{form} = [{\sc nom, nam}], \texttt{dcl} = [{\sc true, false}]\footnote{\texttt{dcl} is almost always {\sc false}}, \texttt{countable} = [{\sc true, false}])
\\
\\
{\bf Path Tag}: (\texttt{id} = [\texttt{p0, p1, p2, }\ldots], \texttt{dimensionality} = [{\sc area, point, line, volume}], \texttt{form} = [{\sc nom, nam}], \texttt{dcl} = [{\sc true, false}, \texttt{countable} = [{\sc true, false}])
\\
\\
{\bf Spatial Entity Tag}: (\texttt{id} = [\texttt{se0, se1, se2, }\ldots], \texttt{dimensionality} = [{\sc area, point, line, volume}], \texttt{form} = [{\sc nom, nam}], \texttt{dcl} = [{\sc true, false}], \texttt{countable} = [{\sc true, false}])
\\
\\
{\bf Motion Tag}: (\texttt{id} = [\texttt{m0, m1, m2 }\ldots], \texttt{motion\_type} = [{\sc compound, manner, path}],\\\texttt{motion\_class} = [{\sc move, move\_external, move\_internal, leave, reach, cross, detach, hit, follow, deviate, stay}], \texttt{motion\_sense} = [{\sc literal, fictive, intrinsic\_change}])
\\
\\
{\bf Spatial Signal Tag}: (\texttt{id} = [\texttt{s0, s1, s2, }\ldots], \texttt{semantic\_type} = [{\sc topological, directional, dir\_top}])
\\
\\
{\bf Motion Signal Tag}: (\texttt{id} = [\texttt{ms0, ms1, ms2, }\ldots])
\\
\\
{\bf Measure Tag}: (\texttt{id} = [\texttt{me0, me1, me2, }\ldots], \texttt{value} = [\emph{NEAR, lte530, few, }\ldots], \texttt{unit} = [\texttt{meters, kilometers, steps, }\ldots)
\\
\\
{\bf Qualitative Link (QSLINK)}: (\texttt{id} = [\texttt{qsl0, qsl1, qsl2, }\ldots], \texttt{fromID}\footnote{\texttt{fromID} denotes the figure} = [\texttt{pl0, se5, p3, }\ldots], \texttt{toID}\footnote{\texttt{toID} denotes the ground} = [\texttt{pl7, se13, p9, }\ldots], \texttt{relType} = [{\sc rcc8}+], \texttt{trigger} = [\texttt{s0, s1, s2, }\ldots])
\\
\\
{\bf Orientation Link (OLINK)}: (\texttt{id} = [\texttt{ol0, ol1, ol2, }\ldots], \texttt{fromID} = [\texttt{pl0, se5, p3, }\ldots], \texttt{toID} = [\texttt{pl7, se13, p9, }\ldots], \texttt{relType} = [{\sc toward, southwest, }\ldots], \texttt{trigger} = [\texttt{s0, s1, s2, }\ldots], \texttt{frame\_type} = [{\sc relative, absolute, intrinsic}], \texttt{referencePt} = [{\sc viewer, southwest, }\ldots], \texttt{projective} = [{\sc true, false}])
\\
\\
{\bf Move Link (MOVELINK)}: (\texttt{id} = [\texttt{mvl0, mvl1, mvl2, }\ldots], \texttt{fromID}\footnote{\texttt{fromID} denotes the motion} = [\texttt{m0, m1, m2, }\ldots], \texttt{toID}\footnote{\texttt{toID} denotes the mover} = [\texttt{se0, se1, se2, }\ldots], \texttt{trigger} = [\texttt{m0, m1, m2, } \ldots], \texttt{source} = [\texttt{p12, pl35, }\ldots], \texttt{goal} = [\texttt{pl10, p9, }\ldots], \texttt{midPoint} = [\texttt{p9, pl10, }\ldots], \texttt{goal} = [\texttt{pl10, p9, }\ldots], \texttt{goal\_reached} = [{\sc yes, uncertain, no}], \texttt{pathID} = [\texttt{p0, p1, p2, }\ldots], \texttt{motion\_signal} = [\texttt{ms0, ms1, ms2, }\ldots])
\newpage
$_{}$
\\
\\
{\bf Measure Link (MEASURELINK)}: (\texttt{id} = [\texttt{ml0, ml1, ml2, }\ldots], \texttt{fromID}\footnote{denotes the extent being measured (from)} = [\texttt{m0, pl17, p5, }\ldots], \texttt{toID}\footnote{denotes the extent being measured (to)} = [\texttt{p8, pl9, m22, }\ldots], \texttt{retype} = [{\sc distance, width, length, height, general\_dimension}], \texttt{val} = [\texttt{me0, me1, me2, }\ldots])
\\
\\
{\bf Meta Link (METALINK)}: (\texttt{id} = [\texttt{metal0, metal1, metal2, }\ldots], \texttt{fromID}  = [\texttt{pl2, p5, se6, } \ldots], \texttt{toID} = [\texttt{pl0, p4, se1, }\ldots], \texttt{relType} = [{\sc coreference, subcoreference, splitcoreference}])
\\
\\
{\large {\bf 3.0 SPRL and ISO-Space}}
\\
\\
Given the tag and link sets for each mark up language, is there a way to map SPRL to ISO-Space?  It appears that a direct, deterministic mapping without machine learning and/or human annotation is not possible.  
\\
\\
For the SPRL tag {\sc trajector}, there is not a clear ISO-Space equivalent.  This is because ISO-Space classifies extents based on what semantic class they belong to, and not necessarily their semantic role in a spatial relation.   In ISO-Space, places, entities, paths, motions, and events can all be trajectors.  Thus in order to turn each {\sc trajector} into an ISO-Space tag, a classifier will be needed to determine this (or human annotation).  However, because the {\sc trajector} refers to a well defined semantic role, we always know what will be the figure when creating an ISO-Space link, at least for {\sc qslink} and {\sc olink}.  ISO-Space tags themselves do not indicate what their semantic role is.  In general it appears a large amount of {\sc trajector}s are {\sc spatial\_entities}: from CP.gold, $37.74\%$ of {\sc trajector}s have \emph{we, us, I, you} as their extents.  Other statistics include: $14.28\%$ of {\sc trajector}s are non-consuming and $9.65\%$ are non-animate pronouns \emph{which, that, where, it}.  Because SPRL does not have any {\sc metalink}s, dereferencing these pronouns will be quite tricky.  The same issues also apply for the {\sc landmark} tag, since it is just a semantic role for a spatial relationship.  However, a quick scanning of the most frequent {\sc landmark} tokens seems to suggest most {\sc landmark}s are ISO-Space {\sc place}s or {\sc path}s.  The first 60 most frequent tags are themselves places/paths, which is $50.09\%$ of the {\sc landmark}s found in CP.gold.  
\\
\\
The SPRL {\sc path} tag, however, seems to directly correspond to ISO-Space's notion of paths, and thus has a direct equivalent in the ISO-Space {\sc path}.  The same is also true for the {\sc spatial\_indicator} tag, which seems to have a direct correspondence with the ISO-Space {\sc spatial\_signal} tag.  
\\
\\
The SPRL {\sc motion\_indicator} tag, however, is very confused.  In some instances it is actually a verb and thus is like the ISO-Space {\sc motion} tag: at least $68.7\%$ of SPRL {\sc motion\_indicator} tags are actually verbs or verb phrases: \emph{reached}, \emph{took the road}, \emph{entered}, etc.  In other instances it is more like the ISO-Space {\sc motion\_signal} tag. 
\\
\\
Finally, the SPRL {\sc distance} tag seems to correspond to the ISO-Space {\sc measure} tag.  However, by itself it is devoid of attributes; information from an SPRL {\sc relation} link would be needed to know if the {\sc distance} is quantitative or qualitative.  SPRL also has a {\sc direction} tag, which has no ISO-Space equivalent tag.  It simply denotes an absolute or relative direction, and should be relatively easy to translate into an ISO-Space tag set.
\\
\\
{\large {\bf 3.1 Extent Classification}}
\\
\\
The most difficult task of extent translation is deciding when an SPRL {\sc trajector} or {\sc landmark} is either an ISO-Space {\sc place}, {\sc path}, {\sc spatial\_entity}, {\sc motion}, or {\sc event}.  I decide to reduce this five class problem to three classes: determining whether one of these SPRL tags is a place, path, or spatial entity.  An exhaustive glance at all such {\sc trajector}s (a total of 170 unique ones) show that none of them could possibly be an ISO-Space {\sc motion} or {\sc event}.  To perform the classification, one may make the assumption that the ISO-Space place, path, and spatial entity tags all form their own homogenous semantic based class.  Thus, a basic word similarity metric that takes into account semantic similarity and distance can be used to gauge whether a given word is a place, path, or spatial entity.  For this task, the Wu-Palmer similarity metric (WUP) (Wu and Palmer, 1994) is used, due to its simplicity in implementation and in testing my hypotheses.  
\\
\\
WUP is readily implemented in NLTK's WordNet package, merely requiring two synsets to compute their probability, unlike other more advanced metrics, such as the Resnik similarity (Resnik), which require corpora and distributions, in addition to a semantic thesaurus like WordNet.  The idea is this: a {\sc trajector} or {\sc landmark} is classified into either a {\sc spatial\_entity}, {\sc path}, or {\sc place} based on which set of corresponding ISO-Space extents maximize the similarity score.  The similarity score is computed as the average similarity of the SPRL extent to all of the unique ISO-Space extents for each possibly label.  As this is meant to be very simple, all words are put in lowercase.  Extents which are made up of multiple words (common in SPRL, but less so in ISO-Space, where the head word is generally only tagged) are split by white space, and the average is computed for each such word, then macro averaged together.  Pronouns are substituted with their corresponding semantic type, e.g. animate pronouns are treated as \emph{person}, locative pronouns (i.e. \emph{there}, \emph{where}) as \emph{place}, and ambiguous pronouns such as \emph{it} are treated as \emph{thing}.  The synset chosen for each word is the first in the list provided by WordNet (which may be the most common synset for that lemma) and all words are treated as nouns.  
\\
\\
The results are surprisingly accurate (see text files for {\sc trajector} and {\sc landmark} correspondingly) given this very basic metric for classification.  The most troublesome extents are those which are proper nouns and denote places or paths--the classifier almost always classifies these as spatial entities, but only because that is the default case.  However, if the proper name is accompanied with identifying information, then it is (usually) properly classified, e.g. \emph{the river Crisul Repede} is properly classified as an ISO-Space {\sc path}, but {\emph Bogdana} is classified incorrectly as an ISO-Space {\sc spatial\_entity}.  There are some inconsistencies as well:\emph{home of Dracula} is a {\sc place}, but \emph{the Dracula hotel} is a {\sc spatial\_entity}.  The classification seems to have picked up highway names: \emph{the E 70} is a {\sc place}, but has no idea how to handle coordinates: \emph{48n 8e} is a {\sc spatial\_entity}.
\\
\\
In an attempt to add more information, I attempted to change the algorithm by making it greedy: instead of (arbitrarily) choosing the first synset of each word, the algorithm would choose the two synsets of any given word pair which maximized the similarity score.  This proved far worse than the algorithm described previously; besides being completely intractable, it performed terribly at classification.  
\\
\\
Given these results, an augmented metric that incorporates distributional information (e.g. the context of the SPRL extent and the contexts of the ISO-Space extents) would probably improve results further.  Handling proper names would be a little trickier: this would involve first identifying one (i.e. a capitalized word) and then determining if it denotes a person or a place (context would probably help resolve this, again consulting information from ISO-Space annotations could be useful for this disambiguation).  Edge cases like coordinates could also be handled.  
\\
\\
In the more general problem, i.e. generating ISO-Space tasks from text automatically, the initial results here could prove useful in identifying tokens which might be involved in a spatial relation, and minimally classifying them as places/paths/spatial entities.  This brings another question: are all paths and places in a text always involved in a spatial relation?  Classifying the remaining extents can use the same method here, with most likely better results, since the remaining SPRL extents are quite homogenous.  The same metric can also be used to guess what the attributes of a given extent might be, e.g. countable or not countable.  

\end{document}